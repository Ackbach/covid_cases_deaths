\documentclass[12pt]{article}
\usepackage{Homework}
\usepackage{xr}
\newcommand{\lineclear}{\vspace{4pt}}
\newcommand{\dhl}{\\ \hline\hline}
\newcommand{\el}{\\ \hline}
\renewcommand{\line}[1]{\overset{\longleftrightarrow}{#1}}
\newcommand{\ray}[1]{\overset{\longrightarrow}{#1}}
\begin{document}
\begin{center}
{\bf COVID Cases and Death Correlation}

by Adrian C. Keister, Ph.D.

2020/11/12
\end{center}

It has been argued\footnote{{\it Visualizing the lagged connection between 
COVID-19 cases and deaths in the United States: An animation using per capita 
state-level data (January	22, 2020 – July 8, 2020)}, by Christian C. Testa, 
Nancy Krieger, Jarvis T. Chen, and William P. Hanage. HCPDS Working Paper 
Volume 19, Number 4.} that COVID death rates lag COVID case rates by 
2-8 weeks. The authors present a data visualization claiming to make
clear the lag in time between cases and deaths. However, the data visualization
does not do so. The visualization shows that cases rose for a while, and then
deaths also started to rise somewhat. Unfortunately, the data only goes out
to August 6, 2020, and does not take into account more recent death rates, which
are much lower and do not fit the narrative. 
In the footnoted article above, it should be noted that the authors
cite a WHO paper, which
mentions this finding on page 14. The WHO paper
says, ``Data on the {\bf progression of disease} is available from a limited
number of reported hospitalized cases (Figure 5).'' - p. 13, emphasis
original.\footnote{{\it Report of the
		WHO-China Join Mission on Coronavirus
		Disease 2019 (COVID-19)}.} Hence the WHO paper is reasoning from actual
	fatalities. 
	
This paper seeks to investigate the claim, to determine its strength. 
On its face, this might seem plausible, since, reasoning as above, anecdotally,
the median time from the onset of symptoms to death is in that time frame.
However, reasoning from single cases to the aggregate commits the fallacy of
composition and is not borne out by the data. It is, in effect,
a highly biased sample, based solely on fatalities, whereas cases involve
many recoveries. 

To show this lack of correlation, I used data from the covidtracking 
project,\footnote{https://covidtracking.com} 
as it seems less biased than other sources. Indeed, one
great hindrance to doing anything with COVID data is the highly politicized
nature of it. I downloaded the all-state data and performed the following
operations:
\begin{enumerate}[{\bf 1.}]
\item Eliminated all columns but date, state, positive, and deathConfirmed.
The positive and deathConfirmed columns correspond to cases and deaths,
respectively. It is important to note that these are cumulative numbers.
\item As reporting is highly irregular on the weekends, I smoothed out the
cumulative numbers by interpolating over the weekend: Saturday's number is
$2/3$ Friday's number plus $1/3$ Monday's number, and Sunday is similar but
weighted more towards Monday. 
\item Then I took a discrete difference in the cases and deaths to produce
daily rates.
\item Finally, I used a series of correlations to determine which correlation
was the strongest, and where it occurred. The idea here is to shift one time
series past the other one day at a time, compute the correlation at each step,
and find the largest such correlation. This should correspond to the lag time,
if there is a reasonably strong correlation. 
\end{enumerate} 

\end{document}






































